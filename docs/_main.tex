% Options for packages loaded elsewhere
\PassOptionsToPackage{unicode}{hyperref}
\PassOptionsToPackage{hyphens}{url}
%
\documentclass[
]{book}
\usepackage{amsmath,amssymb}
\usepackage{lmodern}
\usepackage{ifxetex,ifluatex}
\ifnum 0\ifxetex 1\fi\ifluatex 1\fi=0 % if pdftex
  \usepackage[T1]{fontenc}
  \usepackage[utf8]{inputenc}
  \usepackage{textcomp} % provide euro and other symbols
\else % if luatex or xetex
  \usepackage{unicode-math}
  \defaultfontfeatures{Scale=MatchLowercase}
  \defaultfontfeatures[\rmfamily]{Ligatures=TeX,Scale=1}
\fi
% Use upquote if available, for straight quotes in verbatim environments
\IfFileExists{upquote.sty}{\usepackage{upquote}}{}
\IfFileExists{microtype.sty}{% use microtype if available
  \usepackage[]{microtype}
  \UseMicrotypeSet[protrusion]{basicmath} % disable protrusion for tt fonts
}{}
\makeatletter
\@ifundefined{KOMAClassName}{% if non-KOMA class
  \IfFileExists{parskip.sty}{%
    \usepackage{parskip}
  }{% else
    \setlength{\parindent}{0pt}
    \setlength{\parskip}{6pt plus 2pt minus 1pt}}
}{% if KOMA class
  \KOMAoptions{parskip=half}}
\makeatother
\usepackage{xcolor}
\IfFileExists{xurl.sty}{\usepackage{xurl}}{} % add URL line breaks if available
\IfFileExists{bookmark.sty}{\usepackage{bookmark}}{\usepackage{hyperref}}
\hypersetup{
  pdftitle={Species distribution models (SDM)},
  pdfauthor={AZTI},
  hidelinks,
  pdfcreator={LaTeX via pandoc}}
\urlstyle{same} % disable monospaced font for URLs
\usepackage{longtable,booktabs,array}
\usepackage{calc} % for calculating minipage widths
% Correct order of tables after \paragraph or \subparagraph
\usepackage{etoolbox}
\makeatletter
\patchcmd\longtable{\par}{\if@noskipsec\mbox{}\fi\par}{}{}
\makeatother
% Allow footnotes in longtable head/foot
\IfFileExists{footnotehyper.sty}{\usepackage{footnotehyper}}{\usepackage{footnote}}
\makesavenoteenv{longtable}
\usepackage{graphicx}
\makeatletter
\def\maxwidth{\ifdim\Gin@nat@width>\linewidth\linewidth\else\Gin@nat@width\fi}
\def\maxheight{\ifdim\Gin@nat@height>\textheight\textheight\else\Gin@nat@height\fi}
\makeatother
% Scale images if necessary, so that they will not overflow the page
% margins by default, and it is still possible to overwrite the defaults
% using explicit options in \includegraphics[width, height, ...]{}
\setkeys{Gin}{width=\maxwidth,height=\maxheight,keepaspectratio}
% Set default figure placement to htbp
\makeatletter
\def\fps@figure{htbp}
\makeatother
\setlength{\emergencystretch}{3em} % prevent overfull lines
\providecommand{\tightlist}{%
  \setlength{\itemsep}{0pt}\setlength{\parskip}{0pt}}
\setcounter{secnumdepth}{5}
\usepackage{booktabs}
\ifluatex
  \usepackage{selnolig}  % disable illegal ligatures
\fi
\usepackage[]{natbib}
\bibliographystyle{plainnat}

\title{Species distribution models (SDM)}
\author{AZTI}
\date{2022-05-12}

\begin{document}
\maketitle

{
\setcounter{tocdepth}{1}
\tableofcontents
}
\hypertarget{about}{%
\chapter{About}\label{about}}

This is a short tutorial for constructing species distribution models in R. It describes the whole process from downloading OBIS and GBIF data, generating pseudo-absence data, including environmental data, fitting the model, validating the model and generating the resulting maps for visualization.

The code is available in \href{https://github.com/Fundacion-AZTI/SDM}{AZTI's github repository} repository and the book is readily available \href{https://fundacion-azti.github.io/SDM/}{here}.

\hypertarget{introduction}{%
\chapter{Introduction}\label{introduction}}

Some introduction about SDMs

\hypertarget{data-preparation}{%
\chapter{Data preparation}\label{data-preparation}}

Bla bla bla

\hypertarget{download-presence-data}{%
\section{Download presence data}\label{download-presence-data}}

\hypertarget{create-pseudo-absence-data}{%
\section{Create pseudo-absence data}\label{create-pseudo-absence-data}}

\hypertarget{download-environmental-data}{%
\section{Download environmental data}\label{download-environmental-data}}

\hypertarget{final-data-set}{%
\section{Final data set}\label{final-data-set}}

\hypertarget{model-fit}{%
\chapter{Model fit}\label{model-fit}}

Some introduction about SDMs

\hypertarget{generalised-linear-models}{%
\section{Generalised linear models}\label{generalised-linear-models}}

\hypertarget{generalised-additive-models}{%
\section{Generalised additive models}\label{generalised-additive-models}}

\hypertarget{shape-constrained-generalised-additive-models}{%
\section{Shape-constrained generalised additive models}\label{shape-constrained-generalised-additive-models}}

One citation is \citep{citores_etal_2020}

\hypertarget{model-selection}{%
\chapter{Model selection}\label{model-selection}}

Bla bla

\hypertarget{model-validation}{%
\chapter{Model validation}\label{model-validation}}

Bla bla

\hypertarget{prediction-and-maps}{%
\chapter{Prediction and maps}\label{prediction-and-maps}}

predict from fitted models and produce maps

  \bibliography{references.bib}

\end{document}
